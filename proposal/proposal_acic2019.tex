%\title{Overleaf Memo Template}
% Using the texMemo package by Rob Oakes
\documentclass[a4paper,11pt]{texMemo}
\usepackage[english]{babel}
\usepackage{graphicx, lipsum}
\usepackage{url}

%% Edit the header section here. To include your
%% own logo, upload a file via the files menu.
\memoto{Organizing Committee, Atlantic Causal Inference Conference}
\memofrom{Prof.~Mark J.~van der Laan}
\memosubject{Workshop proposal: The \texttt{tlverse} software ecosystem for
  causal inference}
\memodate{\today}
\logo{\includegraphics[scale=0.15]{figs/ucberkeleyseal_874_540-converted.pdf}}

\begin{document}
\maketitle

\section{Workshop information}

\textbf{Title:} ``The \texttt{tlverse}: A Software Ecosystem for Causal
Inference via Targeted Learning''

\textbf{Goal:} This will primarily be a software workshop centered around the
new \texttt{tlverse} ecosystem (\url{https://github.com/tlverse}) of \texttt{R}
packages. In order to motivate the use of the software tools, there will be
substantive discussion of both causal inference methodology --- largely in the
Targeted Learning framework --- and applications in both observational studies
and randomized experiments.

\section{Abstract:}

...

\section{Motivations}

...

\section{Organizers}

\subsection{Prof.~Mark J.~van der Laan}

Mark van der Laan, PhD, is Professor of Biostatistics and Statistics at UC
Berkeley. His research interests include statistical methods in genomics and
computational biology, survival analysis, censored data, targeted maximum
likelihood estimation in semiparametric models, causal inference, data-adaptive
loss-based super learning, and multiple testing. His research group developed
loss-based super learning in semiparametric models, based on cross-validation,
as a generic optimal tool for estimation of infinite dimensional parameters,
such as nonparametric density estimation and prediction based on censored and
uncensored data. Building on this super learning methodology, his research group
developed targeted maximum likelihood estimation of a target parameter of the
data-generating distribution in semiparametric models, as a new generic optimal
methodology for statistical inference. Most recently, through funding from the
Bill \& Melinda Gates Foundation, Mark's group has been partly focused on the
development of a centralized, principled set of software tools for causal
inference, the \texttt{tlverse}. Contact: \texttt{laan@berkeley.edu}.

\subsection{Prof.~Alan E.~Hubbard}

Alan Hubbard is Professor of Biostatistics, former head of the Division of
Biostatistics at UC Berkeley, and head of data analytics core at UC Berkeley's
SuperFund research program. His current research interests include causal
inference, variable importance analysis, statistical machine learning,
estimation of and inference for data-adaptive statistical target parameters, and
targeted minimum loss-based estimation. Research in his group is generally
motivated by applications to problems in computational biology, epidemiology,
and precision medicine. Contact: \texttt{hubbard@berkeley.edu}.

\subsection{Dr.~Jeremy R.~Coyle}

Jeremy Coyle is a consulting data scientist and statistical programmer,
currently leading the software development effort that has produced the
\texttt{tlverse} ecosystem of \texttt{R} packages and related software tools.
Jeremy earned his PhD in Biostatistics from UC Berkeley in 2016, primarily under
the supervision of Alan Hubbard. Contact: \texttt{jeremy.coyle@gmail.com}.

\subsection{Nima S.~Hejazi}

Nima is a PhD candidate in Biostatistics with a designated emphasis in
computational and genomic biology at UC Berkeley, where he is co-advised by Mark
van der Laan and Alan Hubbard. Nima is affiliated with the UC Berkeley Center
for Computational Biology, the NIH Biomedical Big Data training program at UC
Berkeley, and the Kaiser Permanente Division of Research. His research interests
span causal inference, nonparametric (data-adaptive) inference and machine
learning, targeted minimum loss-based estimation, survival analysis and censored
data models, statistical computing, reproducible research, and computational
biology. Substantive applications have recently included vaccine efficacy
trials, precision medicine, high-dimensional biology, and epidemiology. Nima is
also passionate about software development for applied statistics, including
software design, automated testing, and reproducible coding practices. Contact:
\texttt{nhejazi@berkeley.edu}.

\subsection{Ivana Malenica}

Ivana is a PhD student in the UC Berkeley Biostatistics Division, working
with Alan Hubbard and Mark van der Laan. She earned her Master's in
Biostatistics and Bachelor's in Mathematics, and spent some time at the
Translational Genomics Research Institute. Some of her prior work includes
mathematical modeling, Bayesian models for allele specific expression and
time-series models for genomics data. Broadly, her research interests span
Machine Learning, Causal Inference, high-dimensional data, and semiparametric
theory. Some of her current work has been centered around active learning, model
selection criterion for dependent data, targeted estimators for parameters of
semi- and nonparametric models (recently, working with the natural mediation
effect) and software development (\texttt{medltmle}, \texttt{sl3}). Contact:
\texttt{imalenica@berkeley.edu}.

\section{Duration}

This will be a 6-hour workshop (entire day), featuring self-contained modules
that build on one another, each introducing a distinct type of causal question,
alongside appropriate statistical methodology and software for implementing
solutions to the problem at hand.

\section{Prior History}

The \texttt{tlverse} software ecosystem represents a relatively recent effort
(about 2 years in the making) in developing a set of software tools for causal
inference, all built around a consistent set of design principles. Although we
have introduced the material in courses taught at UC Berkeley, this will be the
first offering in the 6-hour workshop format.

\end{document}

